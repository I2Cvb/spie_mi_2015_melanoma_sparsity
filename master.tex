%  article.tex (Version 3.3, released 19 January 2008)
%  Article to demonstrate format for SPIE Proceedings
%  Special instructions are included in this file after the
%  symbol %>>>>
%  Numerous commands are commented out, but included to show how
%  to effect various options, e.g., to print page numbers, etc.
%  This LaTeX source file is composed for LaTeX2e.

%  The following commands have been added in the SPIE class
%  file (spie.cls) and will not be understood in other classes:
%  \supit{}, \authorinfo{}, \skiplinehalf, \keywords{}
%  The bibliography style file is called spiebib.bst,
%  which replaces the standard style unstr.bst.

\documentclass[letter]{spie}  %>>> use for US letter paper
%\documentclass[a4paper]{spie}  %>>> use this instead for A4 paper
%\documentclass[nocompress]{spie}  %>>> to avoid compression of citations
% \addtolength{\voffset}{9mm}   %>>> moves text field down
% \renewcommand{\baselinestretch}{1.65}   %>>> 1.65 for double spacing, 1.25 for 1.5 spacing

%% Latex documents that need direct input
%  The following command loads a graphics package to include images
%  in the document. It may be necessary to specify a DVI driver option,
%  e.g., [dvips], but that may be inappropriate for some LaTeX
%  installations.
\usepackage[]{graphicx}

% In order to include files without having a clear page using \include*,
% the newclude package is required
\usepackage{newclude}

% Required for acronyms
% use \acresetall to reset the acroyms counter
% macros=True, allows for calling \myTriger rather than \ac{myTriger}
\usepackage[single=true, macros=true, xspace=true]{acro}

% In SPIE template, biblatex can NOT be used to manage the referencing
%
%\usepackage[style=spiebib, backend=biber]{biblatex}

% Clever cross referencing. Using cleverref, instead of writting
% figure~\ref{...} or equation~\ref{...}, only \cref{...} is required.
% The package interprates the references and introduces the figure, fig.,
% equation, eq., etc keywords. \Cref forces first letter capital.
% >> WARNING: This package needs to be loaded after hyperref, math packages,
%             etc. if used.
%             Cleveref is recomended to load late
\usepackage{hyperref}
\usepackage{cleveref}

% To create random text use lipsum
\usepackage{lipsum}
\usepackage{siunitx}
\DeclareSIUnit\px{px}        % contains the latex packages
\title{Classification of Melanoma Lesions Using Sparse Coded Features and Random Forests}
\author{Mojdeh Rastgoo\supit{a,b}, Guillaume Lemaitre\supit{a,b}, Desire Sidibe\supit{a}, Oliver Morel\supit{a}, Franck Marzani\supit{a} and Rafael Garcia\supit{b}
\skiplinehalf
\supit{a}Universit\'e de Bourgogne, Le2i-UMR CNRS 6306,BP 47870, 21078 Dijon, France;\\
\supit{b}Universitat de Girona, Computer Vision and Robotics Group, Campus Montilivi, Edifici PIV, s/n, 17071 Girona, Spain} 

             % contains the Title and Autor info
%>>>> uncomment following for page numbers
% \pagestyle{plain}    
%>>>> uncomment following to start page numbering at 301 
%\setcounter{page}{301} 

\definechangesauthor[name={sik}, color=blue]{sik}
\definechangesauthor[name={moj}, color=orange]{moj}
\definechangesauthor[name={glm}, color=red]{glm}
\setremarkmarkup{(#2)}
      % contains package and variables init.
%% Acronym definition example using glossaries package
%% \usepackage{acro} is required
%% 
%% For a powerful usage of the acro package look at http://tex.stackexchange.com/questions/135975/how-to-define-an-acronym-by-using-other-acronym-and-print-the-abbreviations-toge

\DeclareAcronym{cad}{
  short = CAD, 
  long = Computer-Aided Diagnosis
}

\DeclareAcronym{rf}{
  short = RF,
  long = Random Forests
}

\DeclareAcronym{se}{
  short = SE,
  long = Sensitivity
}

\DeclareAcronym{sp}{
  short = SP,
  long =  Specificity
}

\DeclareAcronym{sift}{
  short = SIFT,
  long =  Scale-Invariant Feature Transform 
}

\DeclareAcronym{svd}{
  short = SVD,
  long =  Singular Value Decomposition 
}

\DeclareAcronym{acs}{
  short = ACS, 
  long = American Cancer Society
}

\DeclareAcronym{omp}{
  short = OMP,
  long =  Orthogonal Matching Pursuit 
}

\DeclareAcronym{mp}{
  short = MP,
  long =  Matching Pursuit 
}
\DeclareAcronym{adb}{
	short = AdB,
	long = AdaBoost
}
\DeclareAcronym{bow}{
	short = BoW, 
	long = bag of words
}
\DeclareAcronym{svm}{
	short= SVM,
	long = Support Vector Machine
}
\DeclareAcronym{nn}{
	short = NN, 
	long = nearest-neighbor
	}      % contains the acronims

%% Select inputing only one part of the document
%\includeonly{content/intro/intro}   % the file wihtout .tex
%\includeonly{content/other/other_content}

% \addbibresource{./content/lit_review.bib}
% \addbibresource{./content/biblatex-examples.bib}

\begin{document}

\maketitle

\begin{abstract}
\acresetall  % reset the acronyms from the title (if any)
Malignant melanoma is the most dangerous type of skin cancer, yet it is the most treatable kind of cancer, conditioned by its early diagnosis which is a challenging task for clinicians and dermatologists.
In this regard, CAD systems based on machine learning and image processing techniques are developed to differentiate melanoma lesions from benign and dysplastic nevi using dermoscopic images. 
Generally, these frameworks are composed of sequential processes: pre-processing, segmentation, and classification. 
This architecture faces mainly two challenges: (i) each process is complex with the need to tune a set of parameters, and is specific to a given dataset; (ii) the performance of each process depends on the previous one, and the errors are accumulated throughout the framework. 
In this paper, we propose a framework for melanoma classification based on sparse coding which does not rely on any pre-processing or lesion segmentation.
Our framework uses Random Forests classifier and sparse representation of three features: SIFT, Hue and Opponent angle histograms, and RGB intensities. 
The experiments are carried out on the public $PH^{2}$ dataset using a 10-fold cross-validation.
The results show that SIFT sparse-coded feature achieves the highest performance with sensitivity and specificity of 100\% and 90.3\% respectively, with a dictionary size of 800 atoms and a sparsity level of 2. 
Furthermore, the descriptor based on RGB intensities achieves similar results with sensitivity and specificity of 100\% and 71.3\%, respectively for a smaller dictionary size of 100 atoms. 
In conclusion, dictionary learning techniques encode strong structures of dermoscopic images and provide discriminant descriptors.
\end{abstract}

\keywords{Melanoma, Classification, Sparse coding, Random forests, Dermoscopy}


%% Incldue the content without .tex extension
\acresetall  % reset the acronyms from the abstract
\include*{content/intro/description} 
\include*{content/method/method}
\include*{content/results/experiments}
\include*{content/results/results}
\include*{content/conclusion/conclusion}
         % the file wihtout .tex
%\include*{content/other/other}
%
%\acknowledgments     %>>>> equivalent to \section*{ACKNOWLEDGMENTS}
%
%This unnumbered section is used to identify those who have aided the authors in understanding or accomplishing the work presented and to acknowledge sources of funding.
%
%\bibliography{./content/lit_review}   %>>>> bibliography data in report.bib
%\bibliographystyle{spiebib}   %>>>> makes bibtex use spiebib.bst

\end{document}

