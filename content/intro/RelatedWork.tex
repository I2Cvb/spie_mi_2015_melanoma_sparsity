\section{Related Work}
\label{sec:rw}
In the past decade, numerous approaches have been proposed for automated recognition of melanoma lesions.
The developed methods were commonly based on clinical or dermoscopy modality.
These methods follow the usual classification framework of computer vision and consists of four common steps of (i) pre-processing, (ii) segmentation, (iii) feature extraction and finally (iv) classification.
Korotkov~\emph{et~al.}\cite{korotkov2012computerized} and Rastgoo~\emph{et~al.}\cite{rastgoo2015automatic} summarize these methods and their properties.
Unfortunately a fair comparison among the presented methods in the state of the art is not possible, due to lack of benchmark and common datasets~\cite{rastgoo2015automatic,korotkov2012computerized}.
Nevertheless recently, thanks to Mendoncca~\emph{et~al.}\cite{mendoncca2013ph}, a public dataset ( PH$^{2}$ ) was resealed for research purposes.
Section~\ref{sec:dataset} presents a detail description of this dataset.

Subsequently, in this section we emphasize the most recent methods evaluated on the the same dataset.
Barata~\emph{et~al.}\cite{barata2013two,barata2013role}, and ruela~\emph{et~al.}\cite{ruela2013role,ruela2013color} used different subsets of PH$^{2}$ dataset in their works~\cite{mendoncca2013ph}.
Ruela~\emph{et~al.}\cite{ruela2013role,ruela2013color} compared the role of shape and colors for detection of melanoma~vs.~benign and dysplastic lesions using \ac{adb} classifier.
While, concerning the same problem, Barata~\emph{et~al.}\cite{barata2013two,barata2013towards} proposed to use \ac{bow} representation of colors and gradient features.
Their results comparing different classifiers, such as \ac{adb}, kernel \ac{svm}, and k-\ac{nn}, indicated that combination of \ac{bow} representation and k-\ac{nn} classifier achieved the best performance with \ac{se} and \ac{sp} of 100$\%$ and 75$\%$, respectively.