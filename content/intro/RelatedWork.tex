\section{Related Work}
\label{sec:rw}
In the past decade, numerous approaches have been proposed for automated recognition of melanoma lesions.
The developed methods were commonly based on clinical or dermoscopy modality.
These methods follow the usual classification framework of computer vision and consists of four common steps of (i) pre-processing, (ii) segmentation, (iii) feature extraction and finally (iv) classification.
Korotkov~\emph{et~al.}\cite{korotkov2012computerized} and Rastgoo~\emph{et~al.}\cite{rastgoo2015automatic} summarize these methods and their properties.
Unfortunately a fair comparison among the presented methods in the state of the art is not possible, due to lack of benchmark and common datasets~\cite{rastgoo2015automatic,korotkov2012computerized}.
Nevertheless recently, thanks to Mendoncca~\emph{et~al.}\cite{mendoncca2013ph}, a public dataset ( PH$^{2}$ ) was resealed for research purposes.
Section~\ref{sec:exp} presents a detail description of this dataset.

Subsequently, in this section we emphasize on the most recent methods which were evaluated using the the same dataset.
Table~\ref{tab:rw} summarizes these methods.

Barata~\emph{et~al.}\cite{barata2013two,barata2013role}, and ruela~\emph{et~al.}\cite{ruela2013role,ruela2013color} used different subsets of PH$^{2}$ dataset in their works~\cite{mendoncca2013ph}.
Ruela~\emph{et~al.}\cite{ruela2013role,ruela2013color} compared the role of shape and colors for detection of melanoma~vs.~benign and dysplastic lesions using \ac{adb} classifier.
While, concerning the same problem, Barata~\emph{et~al.}\cite{barata2013two} proposed to use \ac{bow} representation of colors and gradient features.
Their results comparing different classifiers, such as \ac{adb}, kernel \ac{svm}, and k-\ac{nn}, indicated that combination of \ac{bow} representation and k-\ac{nn} classifier achieved the best performance with \ac{se} and \ac{sp} of 100$\%$ and 75$\%$, respectively.
The similar classification scheme, \ac{bow} representation, k-\ac{nn} classifier, and histogram of opponent color space histogram, later was used by the same authors for comparing the effects of segmentation~\cite{barata2013towards}.
In this study the authors compared the performance of their proposed system while automatic segmentation or manual segmentation provided by a specialist were used.
The obtained results indicated that manual segmentation provided better performance than automatic segmentation (\ac{se} and \ac{sp} of 98\% and 86\%, respectively).
Random over sampling, in feature space, was used by all the aforementioned methods, in order to deal with imbalance problem. 

PH$^{2}$ dataset was also used by Abuzaghleh~\emph{et~al.}\cite{abuzaghleh2014automated}.
In this study the authors proposed automated recognition system based on color and shape features such as 2-D \ac{fft2}, \ac{dct2}, size and complexity features.
The authors proposed two classification approach: (i) multi-class and (ii) two-level classification.
In both approach \ac{svm} classifier was used.
In later approach in the first level, normal and abnormal lesions were classified, while in the second level, the abnormal lesions were divided into melanoma and dysplastic lesions.

In our previous work, using the same dataset, we compared the effects of various colors, shape and texture features and ensemble approaches for classification of melanoma lesions~\cite{rastgoo2015ensemble}.
In this work the features were extracted from the segmented area and \ac{dos} was used instead of \ac{ros}~\cite{rastgoo2015ensemble}.
Using \ac{rf} ensemble and combination of color and texture features the \ac{se} and \ac{sp} of 94\% and 92\%, was achieved, respectively.


%\begin{table}
\begin{table}
	\caption{Summary of the proposed classification methods using PH$^{2}$ dataset.}
\resizebox{1\textwidth}{!}{
\scriptsize{
\begin{threeparttable}
\begin{tabular}{l c c cc	c   c cc}
\toprule
Ref & Segmentation & features & \multicolumn{2}{c}{Classification} & Balancing & Validation & \multicolumn{2}{c}{Best performance} \\
\cmidrule{4-5}\cmidrule{8-9}
    & 			   & 		  & Classifier & Representation  &           &  & \ac{se} & \ac{sp}\\
\midrule 
\multirow{2}{*}{Ruela~\emph{et~al.}}\cite{ruela2013role} & \multirow{2}{*}{$\checkmark$} & Shape, \acs{fd}\tnote{1} & \acs{adb} & - & \ac{ros} & OvA\tnote{1}  & 92 & 74 \\ \\
\multirow{2}{*}{Ruela~\emph{et~al.}}\cite{ruela2013color} & \multirow{2}{*}{$\checkmark$} & Color statistics & k-\acs{nn}, \acs{adb} & - & \ac{ros} & OvA  & 96 & 83 \\ \\
\multirow{2}{*}{Barata~\emph{et~al.}\cite{barata2013two}} &  \multirow{2}{*}{$\checkmark$} & Opponent histogram & \acs{adb}, \acs{svm} & \acs{bow} & \multirow{2}{*}{\acs{ros}} & 10-fold\tnote{2} & \multirow{2}{*}{100} & \multirow{2}{*}{75}   \\
& & HoG & k-\ac{nn} & - &  & OvA &  &\\ \\

Barata~\emph{et~al.}\cite{barata2013towards} & $\checkmark$ & Opponent histogram & k-\acs{nn} & \acs{bow} & \acs{ros} & 10-fold & 98 & 86  \\ \\


Abuzaghleh~\emph{et~al.}\cite{abuzaghleh2014automated} & $\checkmark$ & \acs{fft2}, \acs{dct2} & \ac{svm} & - & - & - & 97.7 & - \\ \\

\multirow{3}{*}{Rastgoo~\emph{et~al.}\cite{rastgoo2015ensemble}} & \multirow{3}{*}{$\checkmark$} & Shape, color statistics & \multirow{3}{*}{-} & \multirow{2}{*}{\acs{rf}}  & \multirow{3}{*}{\acs{dos}}& \multirow{3}{*}{OvA} & \multirow{3}{*}{94} & \multirow{3}{*}{92}\\
&  & opponent angle and Hue histogram & &   &  &  & & \\
&  & CLBP, GLCM, HoG, Gabor\tnote{3} & & LC\tnote{3} & & & & \\
\bottomrule
\end{tabular}
 \begin{tablenotes}
  \item[1] \acf{fd}, One versus all (OvA).
  \item[1] k-fold corss validation.
  \item[2] Completed Local Binary Pattern (CLBP), Gray-Level Co-occurrence Matrix (GLCM), Histogram of Gradients (HoG), Learner combination (LC).
  
%  \item[2] 24 neighbourhood, rotation invariant, uniform and normalized histogram
%  \item[3] \textit{D} stands for distance in pixel and \textit{G} quantized number of grey levels
  \end{tablenotes}

\end{threeparttable}
}}
\label{tab:rw}
\end{table}


%Barata 1 
%One vs all 
%random over sampling 
%local 10 -fold 
%global one vs all 
%
%
%Barata 3 Bag of Words 
%(25 melanomas and 151 nevi).
%10 fold - classification
%random over sampling 
%
%Barata 2 
%10 fold classification
%random over sampling 
%
%Ruela 1 and 2 
%One vs all 
%random over sampling 
%
%Abuzaghleh 
%75 train, 
%25 test 
%Mel, Dys, Nor
%97.7, 91.3, 90.6 
%No info about balancing 
%
%Rastgoo et al 

