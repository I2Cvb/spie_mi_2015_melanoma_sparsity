% include the figures path relative to the master file
\graphicspath{ {./content/intro/figures/} }

\section{Description}
\label{sec:descr}  % \label{} allows reference to this section
Malignant melanoma is a type of skin cancer and although it accounts for almost 2\% of all skin cancer
cases, it is the deadliest type and causes the vast majority of deaths. 
{\color{red} The incidence of melanoma has increased in the past decades and according to \textit{World Health Organization}, annually 132,000 melanoma cases occur globally.}
Only in United states the incidence of melanoma has increased 15 times in the last 40 years which is the most rapid increase among all the cancers. 
In 2014, \textit{American Cancer Society} reported the estimated number of deaths as 9710 individuals.
Nevertheless during the same time, there has been a significant rise in patients survival, thanks to early diagnosis and treatment of melanoma. 
{\color{red} Nevertheless, melanoma is the most treatable kind of cancer, if it is diagnosed early.} 
The clinical prognosis of early stage of melanoma is based on ``ABCDE'' rule, where Asymmetry, irregular Borders, variegated Colors, Diameter greater than \SI{6}{\milli \meter} and Evolving stages over time of the lesions are visually inspected in each clinical routine. 
The inspection is performed using different imaging techniques such as dermoscopy. Visual inspection, similarity between the lesions and the necessity to perform patients follow-up over years makes this task difficult and more prone to errors. 
Thus \ac{cad} systems based on machine learning and image processing techniques have been proposed to assist the dermatologists and clinicians. 
The proposed algorithms generally attend to mimic the characteristics of ``ABCDE'' rule and consist of common steps of pre-processing, segmentation and classification of extracted features. 
This sequential architecture is complex and dependent on individual dataset. 
The aim of this research is to design a more genera system which does not require pre-processing and segmentation of the lesions based on sparse coded features and \ac{rf} classifier.
{\color{red} In this paper we propose a more general framework which does not necessitate pre-processing and segmentation of the lesions based on sparse coded features and random forests classifier.}


% Some stuff that emac's colegues use
%%% Local Variables:
%%% mode: late
%%% TeX-master: "../../master.tex"
%%% End: \section{introduction}

