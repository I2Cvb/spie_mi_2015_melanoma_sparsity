\section{Conclusion and Future Work}
\label{sec:con}
In this work we presented a novel classification framework of melanoma lesions, based on sparse representation of the features. 
Our framework does not need the primary steps of pre-processing and segmentation of the lesions and provide more general algorithm to solve this problem. 
We proposed to use a well-known color descriptor based on hue and opponent color space histograms and SIFT as a texture descriptor. We also consider to represent the images in their simplest form and consider the second color descriptor as R,G and B intensity values from the image.
An extensive comparison based on different dictionary sizes and several sparsity levels were made. 
These comparison was evaluated on $PH^{2}$ dataset. 
The obtain results highlight the advantage of the proposed method. 
where \ac{rf} classifier and sparse representation of SIFT features with a dictionary size of 800 and sparsity level of 2 achieve the highest \ac{se} and \ac{sp} of 100\% and 90.3\%, respectively.
In future, we would make a comparison of sparse learned dictionary with Bag of word approach and would like to extend our dataset. \\


\textbf{Note. This work has not been submitted for publication or presentation elsewhere}

%%% Local Variables: 
%%% mode: latex
%%% TeX-master: "../../master"
%%% End: 