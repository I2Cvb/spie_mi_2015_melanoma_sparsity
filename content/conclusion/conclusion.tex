\section{Conclusion and Future Work}
\label{sec:con}
In this work, we proposed a novel classification framework of melanoma lesions, based on sparse representation of the low-level features. 
Our framework does not rely on the primary steps of pre-processing and segmentation of the lesions and provide more general algorithm to solve this problem. 
We proposed to use a well-known color descriptor based on hue and angle histograms of opponent color space as well as \ac{sift} as a texture descriptor. 
We also consider to represent the images in their simplest form and consider the second color descriptor as R,G and B intensities.
An extensive comparison based on different dictionary sizes and several sparsity levels were carried out on the PH$^{2}$ dataset. 
The results highlighted the advantage of the proposed method where an \ac{rf} classifier and a sparse representation of \ac{sift} features with a dictionary size of 800 and a sparsity level of 2 achieved the highest performance (\ac{se} and \ac{sp} of 100\% and 90.3\%, respectively).
In general, the obtained results outperform the state of the art.
As avenues for future research, a comparison of sparse learned dictionary with Bag-of-Word models can be performed.


%\textbf{Note. This work has not been submitted for publication or presentation elsewhere}

%%% Local Variables: 
%%% mode: latex
%%% TeX-master: "../../master"
%%% End: 