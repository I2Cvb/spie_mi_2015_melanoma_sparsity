%% Acronym definition example using glossaries package
%% \usepackage{acro} is required
%% 
%% For a powerful usage of the acro package look at http://tex.stackexchange.com/questions/135975/how-to-define-an-acronym-by-using-other-acronym-and-print-the-abbreviations-toge

\DeclareAcronym{cad}{
  short = CAD, 
  long = Computer-Aided Diagnosis
}

\DeclareAcronym{rf}{
  short = RF,
  long = Random Forests
}

\DeclareAcronym{se}{
  short = SE,
  long = Sensitivity
}

\DeclareAcronym{sp}{
  short = SP,
  long =  Specificity
}

\DeclareAcronym{sift}{
  short = SIFT,
  long =  Scale-Invariant Feature Transform 
}

\DeclareAcronym{svd}{
  short = SVD,
  long =  Singular Value Decomposition 
}

\DeclareAcronym{acs}{
  short = ACS, 
  long = American Cancer Society
}

\DeclareAcronym{omp}{
  short = OMP,
  long =  Orthogonal Matching Pursuit 
}

\DeclareAcronym{mp}{
  short = MP,
  long =  Matching Pursuit 
}
\DeclareAcronym{adb}{
	short = AdB,
	long = AdaBoost
}
\DeclareAcronym{bow}{
	short = BoW, 
	long = bag of words
}
\DeclareAcronym{svm}{
	short= SVM,
	long = Support Vector Machine
}
\DeclareAcronym{nn}{
	short = NN, 
	long = nearest-neighbor
}
\DeclareAcronym{fft2}{
	short = FFT2, 
	long = Fast Fourier Transform features
}
\DeclareAcronym{dct2}{
	short = DCT2,
	long = Discrete Cosine Transform features
}
\DeclareAcronym{dos}{
	short = DOS, 
	long = data space over-sampling
}
\DeclareAcronym{ros}{
	short = ROS, 
	long = random over-sampling
}
\DeclareAcronym{fd}{
	short = FD, 
	long = Fourier descriptor
}
\DeclareAcronym{ml}{
	short = ML, 
	long = Machine Learning
}