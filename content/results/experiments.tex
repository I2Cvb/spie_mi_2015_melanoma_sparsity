\section{Experiments}
\label{sec:exp}
The experiments are conducted on the public PH$^2$ dataset.
This dataset was acquired at \textit{Dermatology Service of Hospital Pedro Hispano, Matosinhos, Portugal}~\cite{mendoncca2013ph} with Tuebinger Mole Analyzer system with a magnification of $20 \times$.
The 8-bits~RGB color dermoscopic images were obtained under the same conditions with a resolution of $\SI{768}{px} \times \SI{560}{px}$. 
This dataset contains 200 dermoscopic images divided into 80 benign, 80 dysplastic and 40 melanoma lesions. 
The lesions are segmented and their histological diagnosis are provided as ground-truth. 
%Figure~\ref{fig:PH2samples} shows three samples of this dataset, representing melanoma, dysplastic, and benign lesion, respectively. 

In our experiments seven images are discarded due to artefacts such as hair occlusions.
Thus, they are conducted on a subset of the dataset consisting of 39 melanoma, 78 benign, and 76 dysplastic lesions.
The patch size used to extract the feature is $\SI{10}{px} \times \SI{10}{px}$.
The three low-level features are sparsely encoded considering three sparsity levels $\lambda=\{2,4,8\}$ and different number of atoms $K = \{100, 200, \cdots, 1000\}$.
The classification is performed in a 10-fold cross-validation model in which 80\% of the data is used for training and 20\% for testing. 
 